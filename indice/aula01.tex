\chapter{Capitulo 1}
\section[introdução]{Visão Geral - Introdução}
  Introduzir o conceito de computadores como um sistema hierárquico. O computador pode ser visto como um sistema formado por conjunto de estruturado de componentes. Cada componente pode ser descrito em termos de sua estrutura e funções internas.

  $ \bullet $ Os principais elementos de um sistema de computação são as unidades central de processamento \textit{(central processing unit - CPU)}, a memória principal, o subsistema de E/S \textit{(entrada e saída)} e os mecanismos de interconexão entre esses componentes. \\
  A CPU, por sua vez, consiste em uma unidade de controle, uma unidade lógica e aritmética \textit{(arithmetic and logic unit)}, registradores internos e mecanismo internos e mecanismos de interconexão.



\subsection{Arquitetura e Organização}
  O termo \textit{'arquitetura de computadores'} refere aos atributos de um sistema que são visiveis ao programador, em outras plavras que tem impacto direto sobre a execução lógica de um programa. O termo \textit{'organização'} referem-se às unidades operacionais e suas interconexões que implementam as especificações da sua arquitetura.


 \subsection{Estrutura e função}
    O ponto chave é o reconhecimento da natureza hierárquica da maioria dos sistemas. Um sistema hierárquico é constituido por um conjunto de subsistemas inter-relacionados, cada qual, possuindo também uma estrutura hieráquica, contendo, em seu nível mais baixo, subsistemas elementares.

    $ \bullet $ \textbf{Estrutura:} o modo como os componentes estão inter-relacionados. \\
    $ \bullet $ \textbf{Função:} a operação de cada componente individual como parte da estrutura. \\

  \textbf{\textit{Função:}} \\

  Tanto a estrutura quanto as funções de um computador são, em sua essência, muito simples. As fuções basicas que um computador pode desempenhar são:

  $ \bullet $ \textbf{Processamento de Dados} \\
  $ \bullet $ \textbf{Armazenamento de Dados} \\
  $ \bullet $ \textbf{Transferência de Dados} \\
  $ \bullet $ \textbf{Controle} \\


  % Palavra ou símbolo, que exprime um pensamento completo e afirma um fato ou exprime um juizo.
  % Frases declarativas, que podemos jugar como sendo \textit{verdadeiras} ou \textit{falsas}.


  % \begin{tcolorbox}[colback=black!5!white,colframe=black!75!white,title=\textbf{Exemplo}]
  %     Fortaleza é a capital do Ceará. \(\rightarrow (V)\)
  % \tcblower
  %     \(\pi > 4 \rightarrow (F)\)
  % \end{tcolorbox}
  %
  % \textbf{Principios:}
  % \begin{enumerate}
  %     \item \textbf{Não contradição:} \\
  %     Uma proposição não pode ser \textit{V} ou \textit{F} ao mesmo
  %     tempo.
  %     \item \textbf{do terceiro excluído:} \\
  %     Um proposição ou é \textit{V} ou \textit{F} e não há um terceiro termo.
  % \end{enumerate}
  %
  % \begin{tcolorbox}[colback=black!5!white,colframe=black!75!white,title=\textbf{Exemplo}]
  %     Dante escreveu Lusiadas \( \rightarrow (F) \) \\
  %     \textbf{Nessa preposição não há possibilidade de um outra afirmação.}
  %     \tcblower
  %     \( \pi \) \textit{é racional \( \frac{3}{5} \) é um número lógico das proposições.}
  % \end{tcolorbox}
  %
  %
  % \subsection{Tipos de proposição}
  % \textbf{Proposição simples ou composta}
  % \begin{enumerate}
  %         \item \textbf{Simples ou Atômica} \\
  %         São as que contém apenas uma proposição. São representadas pelas letras (p,q,r,s) sempre minusculas
  %         \begin{tcolorbox}[colback=black!5!white,colframe=black!75!white,title=\textbf{Exemplo}]
  %             p: Pedro é Cabeludo \\
  %             q: O número 16 é quadrado perfeito.
  %         \end{tcolorbox}
  %
  %         \item \textbf{Compostas ou Molecular:} \\
  %         São as que contém mais de uma proposição e são representadas pelas letras (P,Q,R,S) sempre maiuscula.
  % \end{enumerate}
  %
  % \subsection{Conectivos Lógicos:}
  % \textbf{Palavra que forma novas proposições}
  % \begin{tcolorbox}[colback=black!5!white,colframe=black!75!white,title=\textbf{Exemplo}]
  %   "e", "ou", "não", "se ...então" e "se ... somente se".
  % \end{tcolorbox}
  %
  % \subsection{Tabela Verdade:}
  % \begin{itemize}
  % \item \textbf{Proposição Simples ou Atómica:}
  %
  %   \begin{tabular}{|c|}
  %     \hline
  %     p \\\hline
  %     V \\\hline
  %     F \\\hline
  %   \end{tabular}
  %
  %   \item \textbf{Proposição Composta ou Molecular}
  %
  %   \begin{tabular}{|c|c|}
  %     \hline
  %     P & Q \\\hline
  %     V & V \\\hline
  %     V & F \\\hline
  %     F & V \\\hline
  %     F & F \\\hline
  %   \end{tabular}
  % \end{itemize}
  %
  % \begin{tcolorbox}[colback=black!5!white,colframe=black!75!white,title=\textbf{Calculo Geral}]
  %     \textbf{Formula \( \rightarrow \) 2 elevado ao número de proposições ou seja \( 2^n \)}
  % \end{tcolorbox}
