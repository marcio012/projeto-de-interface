\section{Visão Geral - Introdução}
\hspace{2.0cm}

Este documento especifica os requisitos do \textbf{\myProject},
apresentando aos usuários as características das soluções
propostas para uma futuro desenvolvimento, fazendo assim
um aparato completo dos problemas e soluções, e fornecendo
aos desenvolvedores as informações necessárias para a execução
do projeto e sua implementação, assim como para a realização
dos testes e homologação.


Esta introdução fornece as informações necessárias para fazer
um bom uso deste documento, explicitando seus objetivos e as
convenções que foram adotadas no texto. As demais seções
apresentam a especificação do \textbf{\myProject} e estão 
organizadas como descrito abaixo:

\begin{itemize}
  \item \textbf{Seção 2 - Descrição geral do produto ou serviço:}

    Apresenta uma visão geral do produto ou serviço, 
    caracterizando qual é o seu escopo e descrevendo seus usuários.

  \item \textbf{Seção 3 - Requisitos funcionais:}

    Lista e descreve os requisitos funcionais do produto ou serviço, 
    especificando seus objetivos, funcionalidades, atores e prioridades.

  \item \textbf{Seção 4 - Requisitos não funcionais:}

    Especifica todos os requisitos não funcionais do produto ou serviço, 
    divididos em requisitos de usabilidade, confiabilidade, desempenho, 
    segurança, distribuição, adequação a padrões e requisitos de hardware 
    e software.

  \item \textbf{Seção 5 - Referências:}

    Contém uma lista de referências para outros documentos relacionados.

  \item \textbf{Seção 6 – Aprovação:}

    Contém as assinaturas do analista e do gestor responsáveis pelo projeto, 
    representando o acordo sobre as características do produto ou serviço 
    a ser desenvolvido.

\end{itemize}

\subsection{Público Alvo}

Professores(as), Monitores(as), Secretarios(as), funcionários da Unifor 
que assim necessitar o uso dos projetores de imagem.

\subsection{Convenções, termos e abreviações}

A correta interpretação deste documento exige o conhecimento de algumas 
convenções e termos específicos, que são descrito a seguir.

% Convenções 

\subsubsection{Identificação dos Requisitos} 
\hspace{1.0cm}
  \textbf{RF} – requisito funcional

  \textbf{RNF} – requisito não-funcional

Identificador do requisito é um número, criado sequencialmente, que
determina que aquele requisito é único para um determinado tipo de 
requisito. 

A numeração inicia com o identificador RF001, RF002, RNF001, RNF002, e prossegue 
sendo incrementado 

\subsubsection{Prioridades dos Requisitos}

Para estabelecer a prioridade dos requisitos foram adotadas as 
denominações “essencial”, “importante” e “desejável”. 

\textbf{Essencial} é o requisito sem o qual o sistema não entra em 
funcionamento. Requisitos essenciais são requisitos imprescindíveis, 
que têm que ser implementados impreterivelmente.

\textbf{Importante} é o requisito sem o qual o sistema entra em 
funcionamento, mas de forma não satisfatória. Requisitos importantes 
devem ser implementados, mas, se não forem, o sistema poderá ser 
implantado e usado mesmo assim.

\textbf{Desejável} é o requisito que não compromete as funcionalidades 
básicas do sistema, isto é, o sistema pode funcionar de forma 
satisfatória sem ele. Requisitos desejáveis são requisitos que podem 
ser deixados para versões posteriores do sistema, caso não haja tempo 
hábil para implementá-los na versão que está sendo especificada.


\subsubsection{Preenchimento de informações do documento de requisitos}

Para particularizar esse documento para um determinado projeto, os trechos 
identificados por uma \textbf{<expressão em negrito>} devem ser substituídos 
pelas informações do projeto. As expressões originais explicitam o tipo de 
informação a ser documentada.


\newpage

