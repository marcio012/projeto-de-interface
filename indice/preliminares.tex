\chapter{Preliminares}
\rule{14cm} {0.03cm}

\section{Preliminares}
  
  \begin{flushleft}
  
    \textbf{Os Conjuntos de números usados em álgebra}
  
      São, em geral, subconjuntos de $\mathbb{R}$, o conjunto dos números reais.

    \textbf{Números naturais $\mathbb{N}$}

      São os números empregados em processos de contagem, p. ex., 1,2,3,4,...

    \textbf{Interios $\mathbb{Z}$ }

      Os números para contagem, acrescidos de seus opostos e 0, p. ex., 0,1,2,3,..., -1, -2, -3, ...

    \textbf{Números racionais $\mathbb{Q}$}

      O conjunto de todos os números que podem ser escritos como quociente $\left (\frac{a}{b} \right) $, b $ \ne 0 $, sendo a e b interios, p. ex., $\left (\frac{3}{17} \right) $; $\left (\frac{10}{3} \right)$ ; ...
 
  \end{flushleft}
  
