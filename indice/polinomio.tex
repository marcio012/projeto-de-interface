\chapter{Polinômio}
\rule{14cm} {0.03cm}
\begin{flushleft}
  \section{Polinômio}
    \subsection{Definição}
    
      Um Polinômio é uma expressão que pode ser escrita como um termo ou uma soma de termos da forma $ax_{2}^{n} \, 1$ $x_{2}^{n} \, 2$ ... $ax_{m}^{n} \, \infty$   
      \begin{enumerate}
        \item \textbf{Ex:.} 5, -20, $ \pi $, t, $ 3x^2 $, $ -15x^3y^4 $, $ \frac{3}{2}xy^4zw $, são \textbf{monômios}.

        \item\textbf{Ex:.} $ x + 5 $, $ x^2 - y^2 $, $ 3x^5y^7 $, $ - \sqrt[2]{3x^3z} $, são \textbf{binômios}. 
      
        \item\textbf{Ex:.} $ x + y + 4z $, $ 5x^2 - 3x + 1 $, $ x^3 - y^3 + t^3 $, $ 8xyz - 5x^2y + 20t^3u $, são \textbf{trinômios}.  

      \end{enumerate}

    \subsection{O Grau de um termo}

      O grau do termo de um polinômio é o expoente da vatiável ou, se houver mais de uma variável, a soma dos expoentes das variáveis. 

    \begin{enumerate}
      \item \textbf{Ex:.} $ 3x^8 $ tem grau 8 

      \item \textbf{Ex:.} $ 12xy^2z^2 $ tem grau 5 (soma dos expoente das variáveis $ x^1 y^2 z^2 $)

      \item \textbf{Ex:.} $ \pi $ tem grau 0

    \end{enumerate}

    O grau do polinômio com mais de um termo é o maior dos graus dos termos individuais. 

    \begin{enumerate}
      \item \textbf{Ex:.} $ x^4 + 3x^2 - 250 $ tem grau 4

      \item \textbf{Ex:.} $ x^3y^2 - 30x^4 $ tem grau 5

      \item \textbf{Ex:.} $ 16 - x - x^10 $ tem grau 10

    \end{enumerate}

    Dois ou mais termos são chamdos de semelhantes se são constantes ou se contêm as mesmas variáveis.

    \begin{enumerate}

      \item \textbf{Ex:.} $ 3x $ e $ 5x $ 
      
      \item \textbf{Ex:.} $ -16x^2y $ e $ 2x^2y $

    \end{enumerate}

    \subsection{Adição}

    Soma de dois ou mais polinômio é obtida por combinação de termos semelhantes. A ordem é irrelevante, embora os polinômios tenha que ser escrito de forma crescente. 

    \begin{center}
      \begin{math}
        a_{n}x^n + a_{n -1}x^n-1 + ... + a_{1}x + a_{0} 
      \end{math}
    \end{center}

    \begin{enumerate}
      \item \textbf{Ex:.} $ 5x^3 + 6x^4 - 8x + 2x^2 = 6x^4 + 5x^3 + 2x^2 - 8x $ (grau 4)

      \item \textbf{Ex:.}
        \begin{eqnarray}
          (x^3-3x^2+8x+7)+(-5x^3-12x+3) & = & x^3-3x^2+8x+7-5x^3-12x+3  \nonumber \\
          & = & -4x^3-3x^2-4x+10 \nonumber
        \end{eqnarray}

    \end{enumerate}

\end{flushleft}
