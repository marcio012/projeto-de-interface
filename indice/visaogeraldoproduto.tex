\section{Visão geral do Produto/Serviço}
\hspace{2.0cm}

%Com a constante insatisfação do corpo de funcionários que de alguma forma fazem 
%uso dos projetores, foi observado a necessidade de criar uma solução de modo 
%a "automatizar" o processo de aquisição e devolução dos projetores, como também 
%a adição de função sugeridas, como a possibilidade e locar previamente o projetor
%de forma democrática respeitando as questão prevista para que a reserva aconteça. 

O Sistema tem como sua função principal registra a coleta (capitação dos Data
Show) e devolução dos projetores, como também, cadastrar aqueles que fazem seu uso, criando assim um 
processo automatizado e controlado do uso dos mesmo, antecipando possíveis 
problemas de quebra e perca de acessórios com os cabos de força e os de HDMI, 
e possibilitando o reparo mais rápido pois haveria informação quase que em tempo real.
Dando aos professores um ganho de tempo. 

\hspace{2.0cm}

\subsection{Definições, siglas e abreviações}

\hspace{1.0cm}

  \begin{tabular}{| l | l |}\hline

    Definição, sigla ou abreviação & Descrição \\ \hline
    \mysiglaproject & Abreviação do nome do projeto \\ \hline
    Unifor Online   & Sistema da Unifor  \\ \hline
    IOS             & Sistema Operacional da Appler \\ \hline

  \end{tabular}


  
\subsection{Sistema relacionados e escopo negativo}

\begin{flushleft}
 
\mysiglaproject, sua integração seria em cima do sistema que a Unifor já tem o
"Unifor Online", um extensão da plataforma aproveitando o banco de dados
existente e usando sua persistencia de dados, onde já tem a base de funcionários
cadastrados.


  \textbf{Escopo Atual}
  
  \begin{enumerate}

    \item Um Estação no ponto de controle, "na secretaria", onde fica a parte responsável
pela administração dos projetores, podendo ser um \textbf{PC}, dando a opção de
gerar consulta no local. 
  
  \end{enumerate}


  \textbf{Escopo Futuro}
  
  \begin{enumerate}
  
    \item A instalação de um totem de auto-atendimento, daria agilidade no atendimento. 

    \item Adicionar nas plataformas mobile existentes, IOS e ANDROID, deixando o portátil e de acesso fácil. 
  
  \end{enumerate}

\end{flushleft}


\subsubsection{Sistema relacionados}

  \begin{center} 


    \mysiglaproject - Unifor On line

    Integrando com o mesmo layout, banco de dados da plataforma existente.

    \hspace{2.0cm}
    \hline{}
    \hspace{2.0cm}


    \mysiglaproject - Mobile IOS
    Integração com a mesmo layout e forma existente. 


    \hspace{2.0cm}
    \hline{}
    \hspace{2.0cm}



    \mysiglaproject & Mobile Android - Integração com a mesmo layout e formas existente.

    \hspace{2.0cm}
    \hline{}
    \hspace{2.0cm}
  
  \end{center}

\subsubsection{Escopo Negativo}

\begin{flushleft}
  
  \begin{itemize}
  
    \item O sistema não terá controle de chaves de sala. 

    \item O sistema não disponibilizara inicialmente de reservas de Data Show.

    \item O sistema não gerar cobrança em caso de quebras do Data Show.

  \end{itemize}

\end{flushleft}


\subsection{Premissas e Restrições}

A satisfação esperada do sistema depende claro dos usuários e sua aceitação pois
todo controle ficara na mão dos usuários sendo que praticas antigas notadas e
costumes de entrega tem que ser revisto, prevendo isso é aconselhado um breve
treinamento das partes envolvidas. 

\subsection{Descrição dos Usuários}

Universidade de Fortaleza - Unifor criada pelo então chanceler Edson Queiroz, 
instalada em um campus de 720 mil metros quadrados, onde se encontra uma 
megaestrutura com cerca de 300 salas de aula e mais de 230 laboratórios 
especializados. O campus também é composto por auditórios, salas de vídeo, 
biblioteca, centro de convivência, núcleo de atenção médica, clínica odontológica, 
parque desportivo, teatro, espaço cultural, escritório para a prática jurídica, 
empresas juniores, TV universitária, escola de ensino infantil e fundamental e 
diversos outros núcleos de prática acadêmica e pesquisa.

O corpo docente altamente qualificado, composto de 1.300 professores, com mais de 80\% de mestres e doutores, é responsável pela supervisão de centenas de projetos de pesquisa no domínio científico, tecnológico, artístico e cultural.

\begin{flushleft}


  \begin{itemize}
    
    \item \textbf{Usuários do Sistema}

      Para a operacionalização dos sistema não e exigido muito em conhecimento, por
      se tratar de uma sistema de fácil aprendizado. 

    \item \textbf{Usuário Secretária} Responsável pelo cadastro dos projetores,
      atuoriza a entrega ao usuário professor ou convidado a pegar o data show,
      como também verifica, (chegagem do esquipamento na hora da entrega),
      conclui a entrega do data show, confirma se o usuário convidado tem
      autorização para pegar o data show.

    \item \textbf{Usuário Professor} Faz pedido de um data show, pode ver a
      disponibilidade dos data show, faz a entrega do data show. 

    \item \textbf{Usuário Convidado} Pede autorização para pegar o data show, se o
      mesmo for autorizado, pode fazer a devolução do data show.

  \end{itemize}

\end{flushleft}